\documentclass[11pt]{article}

    \usepackage[breakable]{tcolorbox}
    \usepackage{parskip} % Stop auto-indenting (to mimic markdown behaviour)
    

    % Basic figure setup, for now with no caption control since it's done
    % automatically by Pandoc (which extracts ![](path) syntax from Markdown).
    \usepackage{graphicx}
    % Maintain compatibility with old templates. Remove in nbconvert 6.0
    \let\Oldincludegraphics\includegraphics
    % Ensure that by default, figures have no caption (until we provide a
    % proper Figure object with a Caption API and a way to capture that
    % in the conversion process - todo).
    \usepackage{caption}
    \DeclareCaptionFormat{nocaption}{}
    \captionsetup{format=nocaption,aboveskip=0pt,belowskip=0pt}

    \usepackage{float}
    \floatplacement{figure}{H} % forces figures to be placed at the correct location
    \usepackage{xcolor} % Allow colors to be defined
    \usepackage{enumerate} % Needed for markdown enumerations to work
    \usepackage{geometry} % Used to adjust the document margins
    \usepackage{amsmath} % Equations
    \usepackage{amssymb} % Equations
    \usepackage{textcomp} % defines textquotesingle
    % Hack from http://tex.stackexchange.com/a/47451/13684:
    \AtBeginDocument{%
        \def\PYZsq{\textquotesingle}% Upright quotes in Pygmentized code
    }
    \usepackage{upquote} % Upright quotes for verbatim code
    \usepackage{eurosym} % defines \euro

    \usepackage{iftex}
    \ifPDFTeX
        \usepackage[T1]{fontenc}
        \IfFileExists{alphabeta.sty}{
              \usepackage{alphabeta}
          }{
              \usepackage[mathletters]{ucs}
              \usepackage[utf8x]{inputenc}
          }
    \else
        \usepackage{fontspec}
        \usepackage{unicode-math}
    \fi

    \usepackage{fancyvrb} % verbatim replacement that allows latex
    \usepackage{grffile} % extends the file name processing of package graphics
                         % to support a larger range
    \makeatletter % fix for old versions of grffile with XeLaTeX
    \@ifpackagelater{grffile}{2019/11/01}
    {
      % Do nothing on new versions
    }
    {
      \def\Gread@@xetex#1{%
        \IfFileExists{"\Gin@base".bb}%
        {\Gread@eps{\Gin@base.bb}}%
        {\Gread@@xetex@aux#1}%
      }
    }
    \makeatother
    \usepackage[Export]{adjustbox} % Used to constrain images to a maximum size
    \adjustboxset{max size={0.9\linewidth}{0.9\paperheight}}

    % The hyperref package gives us a pdf with properly built
    % internal navigation ('pdf bookmarks' for the table of contents,
    % internal cross-reference links, web links for URLs, etc.)
    \usepackage{hyperref}
    % The default LaTeX title has an obnoxious amount of whitespace. By default,
    % titling removes some of it. It also provides customization options.
    \usepackage{titling}
    \usepackage{longtable} % longtable support required by pandoc >1.10
    \usepackage{booktabs}  % table support for pandoc > 1.12.2
    \usepackage{array}     % table support for pandoc >= 2.11.3
    \usepackage{calc}      % table minipage width calculation for pandoc >= 2.11.1
    \usepackage[inline]{enumitem} % IRkernel/repr support (it uses the enumerate* environment)
    \usepackage[normalem]{ulem} % ulem is needed to support strikethroughs (\sout)
                                % normalem makes italics be italics, not underlines
    \usepackage{mathrsfs}
    

    
    % Colors for the hyperref package
    \definecolor{urlcolor}{rgb}{0,.145,.698}
    \definecolor{linkcolor}{rgb}{.71,0.21,0.01}
    \definecolor{citecolor}{rgb}{.12,.54,.11}

    % ANSI colors
    \definecolor{ansi-black}{HTML}{3E424D}
    \definecolor{ansi-black-intense}{HTML}{282C36}
    \definecolor{ansi-red}{HTML}{E75C58}
    \definecolor{ansi-red-intense}{HTML}{B22B31}
    \definecolor{ansi-green}{HTML}{00A250}
    \definecolor{ansi-green-intense}{HTML}{007427}
    \definecolor{ansi-yellow}{HTML}{DDB62B}
    \definecolor{ansi-yellow-intense}{HTML}{B27D12}
    \definecolor{ansi-blue}{HTML}{208FFB}
    \definecolor{ansi-blue-intense}{HTML}{0065CA}
    \definecolor{ansi-magenta}{HTML}{D160C4}
    \definecolor{ansi-magenta-intense}{HTML}{A03196}
    \definecolor{ansi-cyan}{HTML}{60C6C8}
    \definecolor{ansi-cyan-intense}{HTML}{258F8F}
    \definecolor{ansi-white}{HTML}{C5C1B4}
    \definecolor{ansi-white-intense}{HTML}{A1A6B2}
    \definecolor{ansi-default-inverse-fg}{HTML}{FFFFFF}
    \definecolor{ansi-default-inverse-bg}{HTML}{000000}

    % common color for the border for error outputs.
    \definecolor{outerrorbackground}{HTML}{FFDFDF}

    % commands and environments needed by pandoc snippets
    % extracted from the output of `pandoc -s`
    \providecommand{\tightlist}{%
      \setlength{\itemsep}{0pt}\setlength{\parskip}{0pt}}
    \DefineVerbatimEnvironment{Highlighting}{Verbatim}{commandchars=\\\{\}}
    % Add ',fontsize=\small' for more characters per line
    \newenvironment{Shaded}{}{}
    \newcommand{\KeywordTok}[1]{\textcolor[rgb]{0.00,0.44,0.13}{\textbf{{#1}}}}
    \newcommand{\DataTypeTok}[1]{\textcolor[rgb]{0.56,0.13,0.00}{{#1}}}
    \newcommand{\DecValTok}[1]{\textcolor[rgb]{0.25,0.63,0.44}{{#1}}}
    \newcommand{\BaseNTok}[1]{\textcolor[rgb]{0.25,0.63,0.44}{{#1}}}
    \newcommand{\FloatTok}[1]{\textcolor[rgb]{0.25,0.63,0.44}{{#1}}}
    \newcommand{\CharTok}[1]{\textcolor[rgb]{0.25,0.44,0.63}{{#1}}}
    \newcommand{\StringTok}[1]{\textcolor[rgb]{0.25,0.44,0.63}{{#1}}}
    \newcommand{\CommentTok}[1]{\textcolor[rgb]{0.38,0.63,0.69}{\textit{{#1}}}}
    \newcommand{\OtherTok}[1]{\textcolor[rgb]{0.00,0.44,0.13}{{#1}}}
    \newcommand{\AlertTok}[1]{\textcolor[rgb]{1.00,0.00,0.00}{\textbf{{#1}}}}
    \newcommand{\FunctionTok}[1]{\textcolor[rgb]{0.02,0.16,0.49}{{#1}}}
    \newcommand{\RegionMarkerTok}[1]{{#1}}
    \newcommand{\ErrorTok}[1]{\textcolor[rgb]{1.00,0.00,0.00}{\textbf{{#1}}}}
    \newcommand{\NormalTok}[1]{{#1}}

    % Additional commands for more recent versions of Pandoc
    \newcommand{\ConstantTok}[1]{\textcolor[rgb]{0.53,0.00,0.00}{{#1}}}
    \newcommand{\SpecialCharTok}[1]{\textcolor[rgb]{0.25,0.44,0.63}{{#1}}}
    \newcommand{\VerbatimStringTok}[1]{\textcolor[rgb]{0.25,0.44,0.63}{{#1}}}
    \newcommand{\SpecialStringTok}[1]{\textcolor[rgb]{0.73,0.40,0.53}{{#1}}}
    \newcommand{\ImportTok}[1]{{#1}}
    \newcommand{\DocumentationTok}[1]{\textcolor[rgb]{0.73,0.13,0.13}{\textit{{#1}}}}
    \newcommand{\AnnotationTok}[1]{\textcolor[rgb]{0.38,0.63,0.69}{\textbf{\textit{{#1}}}}}
    \newcommand{\CommentVarTok}[1]{\textcolor[rgb]{0.38,0.63,0.69}{\textbf{\textit{{#1}}}}}
    \newcommand{\VariableTok}[1]{\textcolor[rgb]{0.10,0.09,0.49}{{#1}}}
    \newcommand{\ControlFlowTok}[1]{\textcolor[rgb]{0.00,0.44,0.13}{\textbf{{#1}}}}
    \newcommand{\OperatorTok}[1]{\textcolor[rgb]{0.40,0.40,0.40}{{#1}}}
    \newcommand{\BuiltInTok}[1]{{#1}}
    \newcommand{\ExtensionTok}[1]{{#1}}
    \newcommand{\PreprocessorTok}[1]{\textcolor[rgb]{0.74,0.48,0.00}{{#1}}}
    \newcommand{\AttributeTok}[1]{\textcolor[rgb]{0.49,0.56,0.16}{{#1}}}
    \newcommand{\InformationTok}[1]{\textcolor[rgb]{0.38,0.63,0.69}{\textbf{\textit{{#1}}}}}
    \newcommand{\WarningTok}[1]{\textcolor[rgb]{0.38,0.63,0.69}{\textbf{\textit{{#1}}}}}


    % Define a nice break command that doesn't care if a line doesn't already
    % exist.
    \def\br{\hspace*{\fill} \\* }
    % Math Jax compatibility definitions
    \def\gt{>}
    \def\lt{<}
    \let\Oldtex\TeX
    \let\Oldlatex\LaTeX
    \renewcommand{\TeX}{\textrm{\Oldtex}}
    \renewcommand{\LaTeX}{\textrm{\Oldlatex}}
    % Document parameters
    % Document title
    \title{백준 알람시계 - a+b-4}
    
    
    
    
    
% Pygments definitions
\makeatletter
\def\PY@reset{\let\PY@it=\relax \let\PY@bf=\relax%
    \let\PY@ul=\relax \let\PY@tc=\relax%
    \let\PY@bc=\relax \let\PY@ff=\relax}
\def\PY@tok#1{\csname PY@tok@#1\endcsname}
\def\PY@toks#1+{\ifx\relax#1\empty\else%
    \PY@tok{#1}\expandafter\PY@toks\fi}
\def\PY@do#1{\PY@bc{\PY@tc{\PY@ul{%
    \PY@it{\PY@bf{\PY@ff{#1}}}}}}}
\def\PY#1#2{\PY@reset\PY@toks#1+\relax+\PY@do{#2}}

\@namedef{PY@tok@w}{\def\PY@tc##1{\textcolor[rgb]{0.73,0.73,0.73}{##1}}}
\@namedef{PY@tok@c}{\let\PY@it=\textit\def\PY@tc##1{\textcolor[rgb]{0.24,0.48,0.48}{##1}}}
\@namedef{PY@tok@cp}{\def\PY@tc##1{\textcolor[rgb]{0.61,0.40,0.00}{##1}}}
\@namedef{PY@tok@k}{\let\PY@bf=\textbf\def\PY@tc##1{\textcolor[rgb]{0.00,0.50,0.00}{##1}}}
\@namedef{PY@tok@kp}{\def\PY@tc##1{\textcolor[rgb]{0.00,0.50,0.00}{##1}}}
\@namedef{PY@tok@kt}{\def\PY@tc##1{\textcolor[rgb]{0.69,0.00,0.25}{##1}}}
\@namedef{PY@tok@o}{\def\PY@tc##1{\textcolor[rgb]{0.40,0.40,0.40}{##1}}}
\@namedef{PY@tok@ow}{\let\PY@bf=\textbf\def\PY@tc##1{\textcolor[rgb]{0.67,0.13,1.00}{##1}}}
\@namedef{PY@tok@nb}{\def\PY@tc##1{\textcolor[rgb]{0.00,0.50,0.00}{##1}}}
\@namedef{PY@tok@nf}{\def\PY@tc##1{\textcolor[rgb]{0.00,0.00,1.00}{##1}}}
\@namedef{PY@tok@nc}{\let\PY@bf=\textbf\def\PY@tc##1{\textcolor[rgb]{0.00,0.00,1.00}{##1}}}
\@namedef{PY@tok@nn}{\let\PY@bf=\textbf\def\PY@tc##1{\textcolor[rgb]{0.00,0.00,1.00}{##1}}}
\@namedef{PY@tok@ne}{\let\PY@bf=\textbf\def\PY@tc##1{\textcolor[rgb]{0.80,0.25,0.22}{##1}}}
\@namedef{PY@tok@nv}{\def\PY@tc##1{\textcolor[rgb]{0.10,0.09,0.49}{##1}}}
\@namedef{PY@tok@no}{\def\PY@tc##1{\textcolor[rgb]{0.53,0.00,0.00}{##1}}}
\@namedef{PY@tok@nl}{\def\PY@tc##1{\textcolor[rgb]{0.46,0.46,0.00}{##1}}}
\@namedef{PY@tok@ni}{\let\PY@bf=\textbf\def\PY@tc##1{\textcolor[rgb]{0.44,0.44,0.44}{##1}}}
\@namedef{PY@tok@na}{\def\PY@tc##1{\textcolor[rgb]{0.41,0.47,0.13}{##1}}}
\@namedef{PY@tok@nt}{\let\PY@bf=\textbf\def\PY@tc##1{\textcolor[rgb]{0.00,0.50,0.00}{##1}}}
\@namedef{PY@tok@nd}{\def\PY@tc##1{\textcolor[rgb]{0.67,0.13,1.00}{##1}}}
\@namedef{PY@tok@s}{\def\PY@tc##1{\textcolor[rgb]{0.73,0.13,0.13}{##1}}}
\@namedef{PY@tok@sd}{\let\PY@it=\textit\def\PY@tc##1{\textcolor[rgb]{0.73,0.13,0.13}{##1}}}
\@namedef{PY@tok@si}{\let\PY@bf=\textbf\def\PY@tc##1{\textcolor[rgb]{0.64,0.35,0.47}{##1}}}
\@namedef{PY@tok@se}{\let\PY@bf=\textbf\def\PY@tc##1{\textcolor[rgb]{0.67,0.36,0.12}{##1}}}
\@namedef{PY@tok@sr}{\def\PY@tc##1{\textcolor[rgb]{0.64,0.35,0.47}{##1}}}
\@namedef{PY@tok@ss}{\def\PY@tc##1{\textcolor[rgb]{0.10,0.09,0.49}{##1}}}
\@namedef{PY@tok@sx}{\def\PY@tc##1{\textcolor[rgb]{0.00,0.50,0.00}{##1}}}
\@namedef{PY@tok@m}{\def\PY@tc##1{\textcolor[rgb]{0.40,0.40,0.40}{##1}}}
\@namedef{PY@tok@gh}{\let\PY@bf=\textbf\def\PY@tc##1{\textcolor[rgb]{0.00,0.00,0.50}{##1}}}
\@namedef{PY@tok@gu}{\let\PY@bf=\textbf\def\PY@tc##1{\textcolor[rgb]{0.50,0.00,0.50}{##1}}}
\@namedef{PY@tok@gd}{\def\PY@tc##1{\textcolor[rgb]{0.63,0.00,0.00}{##1}}}
\@namedef{PY@tok@gi}{\def\PY@tc##1{\textcolor[rgb]{0.00,0.52,0.00}{##1}}}
\@namedef{PY@tok@gr}{\def\PY@tc##1{\textcolor[rgb]{0.89,0.00,0.00}{##1}}}
\@namedef{PY@tok@ge}{\let\PY@it=\textit}
\@namedef{PY@tok@gs}{\let\PY@bf=\textbf}
\@namedef{PY@tok@gp}{\let\PY@bf=\textbf\def\PY@tc##1{\textcolor[rgb]{0.00,0.00,0.50}{##1}}}
\@namedef{PY@tok@go}{\def\PY@tc##1{\textcolor[rgb]{0.44,0.44,0.44}{##1}}}
\@namedef{PY@tok@gt}{\def\PY@tc##1{\textcolor[rgb]{0.00,0.27,0.87}{##1}}}
\@namedef{PY@tok@err}{\def\PY@bc##1{{\setlength{\fboxsep}{\string -\fboxrule}\fcolorbox[rgb]{1.00,0.00,0.00}{1,1,1}{\strut ##1}}}}
\@namedef{PY@tok@kc}{\let\PY@bf=\textbf\def\PY@tc##1{\textcolor[rgb]{0.00,0.50,0.00}{##1}}}
\@namedef{PY@tok@kd}{\let\PY@bf=\textbf\def\PY@tc##1{\textcolor[rgb]{0.00,0.50,0.00}{##1}}}
\@namedef{PY@tok@kn}{\let\PY@bf=\textbf\def\PY@tc##1{\textcolor[rgb]{0.00,0.50,0.00}{##1}}}
\@namedef{PY@tok@kr}{\let\PY@bf=\textbf\def\PY@tc##1{\textcolor[rgb]{0.00,0.50,0.00}{##1}}}
\@namedef{PY@tok@bp}{\def\PY@tc##1{\textcolor[rgb]{0.00,0.50,0.00}{##1}}}
\@namedef{PY@tok@fm}{\def\PY@tc##1{\textcolor[rgb]{0.00,0.00,1.00}{##1}}}
\@namedef{PY@tok@vc}{\def\PY@tc##1{\textcolor[rgb]{0.10,0.09,0.49}{##1}}}
\@namedef{PY@tok@vg}{\def\PY@tc##1{\textcolor[rgb]{0.10,0.09,0.49}{##1}}}
\@namedef{PY@tok@vi}{\def\PY@tc##1{\textcolor[rgb]{0.10,0.09,0.49}{##1}}}
\@namedef{PY@tok@vm}{\def\PY@tc##1{\textcolor[rgb]{0.10,0.09,0.49}{##1}}}
\@namedef{PY@tok@sa}{\def\PY@tc##1{\textcolor[rgb]{0.73,0.13,0.13}{##1}}}
\@namedef{PY@tok@sb}{\def\PY@tc##1{\textcolor[rgb]{0.73,0.13,0.13}{##1}}}
\@namedef{PY@tok@sc}{\def\PY@tc##1{\textcolor[rgb]{0.73,0.13,0.13}{##1}}}
\@namedef{PY@tok@dl}{\def\PY@tc##1{\textcolor[rgb]{0.73,0.13,0.13}{##1}}}
\@namedef{PY@tok@s2}{\def\PY@tc##1{\textcolor[rgb]{0.73,0.13,0.13}{##1}}}
\@namedef{PY@tok@sh}{\def\PY@tc##1{\textcolor[rgb]{0.73,0.13,0.13}{##1}}}
\@namedef{PY@tok@s1}{\def\PY@tc##1{\textcolor[rgb]{0.73,0.13,0.13}{##1}}}
\@namedef{PY@tok@mb}{\def\PY@tc##1{\textcolor[rgb]{0.40,0.40,0.40}{##1}}}
\@namedef{PY@tok@mf}{\def\PY@tc##1{\textcolor[rgb]{0.40,0.40,0.40}{##1}}}
\@namedef{PY@tok@mh}{\def\PY@tc##1{\textcolor[rgb]{0.40,0.40,0.40}{##1}}}
\@namedef{PY@tok@mi}{\def\PY@tc##1{\textcolor[rgb]{0.40,0.40,0.40}{##1}}}
\@namedef{PY@tok@il}{\def\PY@tc##1{\textcolor[rgb]{0.40,0.40,0.40}{##1}}}
\@namedef{PY@tok@mo}{\def\PY@tc##1{\textcolor[rgb]{0.40,0.40,0.40}{##1}}}
\@namedef{PY@tok@ch}{\let\PY@it=\textit\def\PY@tc##1{\textcolor[rgb]{0.24,0.48,0.48}{##1}}}
\@namedef{PY@tok@cm}{\let\PY@it=\textit\def\PY@tc##1{\textcolor[rgb]{0.24,0.48,0.48}{##1}}}
\@namedef{PY@tok@cpf}{\let\PY@it=\textit\def\PY@tc##1{\textcolor[rgb]{0.24,0.48,0.48}{##1}}}
\@namedef{PY@tok@c1}{\let\PY@it=\textit\def\PY@tc##1{\textcolor[rgb]{0.24,0.48,0.48}{##1}}}
\@namedef{PY@tok@cs}{\let\PY@it=\textit\def\PY@tc##1{\textcolor[rgb]{0.24,0.48,0.48}{##1}}}

\def\PYZbs{\char`\\}
\def\PYZus{\char`\_}
\def\PYZob{\char`\{}
\def\PYZcb{\char`\}}
\def\PYZca{\char`\^}
\def\PYZam{\char`\&}
\def\PYZlt{\char`\<}
\def\PYZgt{\char`\>}
\def\PYZsh{\char`\#}
\def\PYZpc{\char`\%}
\def\PYZdl{\char`\$}
\def\PYZhy{\char`\-}
\def\PYZsq{\char`\'}
\def\PYZdq{\char`\"}
\def\PYZti{\char`\~}
% for compatibility with earlier versions
\def\PYZat{@}
\def\PYZlb{[}
\def\PYZrb{]}
\makeatother


    % For linebreaks inside Verbatim environment from package fancyvrb.
    \makeatletter
        \newbox\Wrappedcontinuationbox
        \newbox\Wrappedvisiblespacebox
        \newcommand*\Wrappedvisiblespace {\textcolor{red}{\textvisiblespace}}
        \newcommand*\Wrappedcontinuationsymbol {\textcolor{red}{\llap{\tiny$\m@th\hookrightarrow$}}}
        \newcommand*\Wrappedcontinuationindent {3ex }
        \newcommand*\Wrappedafterbreak {\kern\Wrappedcontinuationindent\copy\Wrappedcontinuationbox}
        % Take advantage of the already applied Pygments mark-up to insert
        % potential linebreaks for TeX processing.
        %        {, <, #, %, $, ' and ": go to next line.
        %        _, }, ^, &, >, - and ~: stay at end of broken line.
        % Use of \textquotesingle for straight quote.
        \newcommand*\Wrappedbreaksatspecials {%
            \def\PYGZus{\discretionary{\char`\_}{\Wrappedafterbreak}{\char`\_}}%
            \def\PYGZob{\discretionary{}{\Wrappedafterbreak\char`\{}{\char`\{}}%
            \def\PYGZcb{\discretionary{\char`\}}{\Wrappedafterbreak}{\char`\}}}%
            \def\PYGZca{\discretionary{\char`\^}{\Wrappedafterbreak}{\char`\^}}%
            \def\PYGZam{\discretionary{\char`\&}{\Wrappedafterbreak}{\char`\&}}%
            \def\PYGZlt{\discretionary{}{\Wrappedafterbreak\char`\<}{\char`\<}}%
            \def\PYGZgt{\discretionary{\char`\>}{\Wrappedafterbreak}{\char`\>}}%
            \def\PYGZsh{\discretionary{}{\Wrappedafterbreak\char`\#}{\char`\#}}%
            \def\PYGZpc{\discretionary{}{\Wrappedafterbreak\char`\%}{\char`\%}}%
            \def\PYGZdl{\discretionary{}{\Wrappedafterbreak\char`\$}{\char`\$}}%
            \def\PYGZhy{\discretionary{\char`\-}{\Wrappedafterbreak}{\char`\-}}%
            \def\PYGZsq{\discretionary{}{\Wrappedafterbreak\textquotesingle}{\textquotesingle}}%
            \def\PYGZdq{\discretionary{}{\Wrappedafterbreak\char`\"}{\char`\"}}%
            \def\PYGZti{\discretionary{\char`\~}{\Wrappedafterbreak}{\char`\~}}%
        }
        % Some characters . , ; ? ! / are not pygmentized.
        % This macro makes them "active" and they will insert potential linebreaks
        \newcommand*\Wrappedbreaksatpunct {%
            \lccode`\~`\.\lowercase{\def~}{\discretionary{\hbox{\char`\.}}{\Wrappedafterbreak}{\hbox{\char`\.}}}%
            \lccode`\~`\,\lowercase{\def~}{\discretionary{\hbox{\char`\,}}{\Wrappedafterbreak}{\hbox{\char`\,}}}%
            \lccode`\~`\;\lowercase{\def~}{\discretionary{\hbox{\char`\;}}{\Wrappedafterbreak}{\hbox{\char`\;}}}%
            \lccode`\~`\:\lowercase{\def~}{\discretionary{\hbox{\char`\:}}{\Wrappedafterbreak}{\hbox{\char`\:}}}%
            \lccode`\~`\?\lowercase{\def~}{\discretionary{\hbox{\char`\?}}{\Wrappedafterbreak}{\hbox{\char`\?}}}%
            \lccode`\~`\!\lowercase{\def~}{\discretionary{\hbox{\char`\!}}{\Wrappedafterbreak}{\hbox{\char`\!}}}%
            \lccode`\~`\/\lowercase{\def~}{\discretionary{\hbox{\char`\/}}{\Wrappedafterbreak}{\hbox{\char`\/}}}%
            \catcode`\.\active
            \catcode`\,\active
            \catcode`\;\active
            \catcode`\:\active
            \catcode`\?\active
            \catcode`\!\active
            \catcode`\/\active
            \lccode`\~`\~
        }
    \makeatother

    \let\OriginalVerbatim=\Verbatim
    \makeatletter
    \renewcommand{\Verbatim}[1][1]{%
        %\parskip\z@skip
        \sbox\Wrappedcontinuationbox {\Wrappedcontinuationsymbol}%
        \sbox\Wrappedvisiblespacebox {\FV@SetupFont\Wrappedvisiblespace}%
        \def\FancyVerbFormatLine ##1{\hsize\linewidth
            \vtop{\raggedright\hyphenpenalty\z@\exhyphenpenalty\z@
                \doublehyphendemerits\z@\finalhyphendemerits\z@
                \strut ##1\strut}%
        }%
        % If the linebreak is at a space, the latter will be displayed as visible
        % space at end of first line, and a continuation symbol starts next line.
        % Stretch/shrink are however usually zero for typewriter font.
        \def\FV@Space {%
            \nobreak\hskip\z@ plus\fontdimen3\font minus\fontdimen4\font
            \discretionary{\copy\Wrappedvisiblespacebox}{\Wrappedafterbreak}
            {\kern\fontdimen2\font}%
        }%

        % Allow breaks at special characters using \PYG... macros.
        \Wrappedbreaksatspecials
        % Breaks at punctuation characters . , ; ? ! and / need catcode=\active
        \OriginalVerbatim[#1,codes*=\Wrappedbreaksatpunct]%
    }
    \makeatother

    % Exact colors from NB
    \definecolor{incolor}{HTML}{303F9F}
    \definecolor{outcolor}{HTML}{D84315}
    \definecolor{cellborder}{HTML}{CFCFCF}
    \definecolor{cellbackground}{HTML}{F7F7F7}

    % prompt
    \makeatletter
    \newcommand{\boxspacing}{\kern\kvtcb@left@rule\kern\kvtcb@boxsep}
    \makeatother
    \newcommand{\prompt}[4]{
        {\ttfamily\llap{{\color{#2}[#3]:\hspace{3pt}#4}}\vspace{-\baselineskip}}
    }
    

    
    % Prevent overflowing lines due to hard-to-break entities
    \sloppy
    % Setup hyperref package
    \hypersetup{
      breaklinks=true,  % so long urls are correctly broken across lines
      colorlinks=true,
      urlcolor=urlcolor,
      linkcolor=linkcolor,
      citecolor=citecolor,
      }
    % Slightly bigger margins than the latex defaults
    
    \geometry{verbose,tmargin=1in,bmargin=1in,lmargin=1in,rmargin=1in}
    
    

\begin{document}
    
    \maketitle
    
    

    
    \begin{tcolorbox}[breakable, size=fbox, boxrule=1pt, pad at break*=1mm,colback=cellbackground, colframe=cellborder]
\prompt{In}{incolor}{ }{\boxspacing}
\begin{Verbatim}[commandchars=\\\{\}]
\PY{c+c1}{\PYZsh{} 백준 2884 알람시계}
\PY{n}{h}\PY{p}{,}\PY{n}{m} \PY{o}{=} \PY{n+nb}{map}\PY{p}{(}\PY{n+nb}{int}\PY{p}{,}\PY{n+nb}{input}\PY{p}{(}\PY{p}{)}\PY{o}{.}\PY{n}{split}\PY{p}{(}\PY{p}{)}\PY{p}{)}
\PY{k}{if} \PY{n}{m} \PY{o}{\PYZlt{}} \PY{l+m+mi}{45} \PY{p}{:}   \PY{c+c1}{\PYZsh{} 45분을 뺴게 되니까 h가 0일 때는 23으로 변경되어야함.}
    \PY{k}{if} \PY{n}{h} \PY{o}{==} \PY{l+m+mi}{0} \PY{p}{:}  \PY{c+c1}{\PYZsh{} = : 할당연산자 / == : a=b (좌측데이터 = 우측데이터 )}
        \PY{n}{h} \PY{o}{=} \PY{l+m+mi}{23}
        \PY{n}{m} \PY{o}{+}\PY{o}{=} \PY{l+m+mi}{60}   \PY{c+c1}{\PYZsh{} m += 60 \PYZhy{}\PYZhy{}\PYZgt{} m = m + 60}
    \PY{k}{else}\PY{p}{:} 
        \PY{n}{h} \PY{o}{\PYZhy{}}\PY{o}{=} \PY{l+m+mi}{1}    \PY{c+c1}{\PYZsh{} h \PYZhy{}= 1 \PYZhy{}\PYZhy{}\PYZgt{} h = h \PYZhy{}1 }
        \PY{n}{m} \PY{o}{+}\PY{o}{=} \PY{l+m+mi}{60}   \PY{c+c1}{\PYZsh{} m = m + 60}
        
\PY{n+nb}{print}\PY{p}{(}\PY{n}{h}\PY{p}{,}\PY{n}{m}\PY{o}{\PYZhy{}}\PY{l+m+mi}{45}\PY{p}{)}
\end{Verbatim}
\end{tcolorbox}

    \begin{tcolorbox}[breakable, size=fbox, boxrule=1pt, pad at break*=1mm,colback=cellbackground, colframe=cellborder]
\prompt{In}{incolor}{ }{\boxspacing}
\begin{Verbatim}[commandchars=\\\{\}]
\PY{c+c1}{\PYZsh{} 백준 2908 상수}
\PY{n}{a}\PY{p}{,}\PY{n}{b} \PY{o}{=} \PY{n+nb}{input}\PY{p}{(}\PY{p}{)}\PY{o}{.}\PY{n}{split}\PY{p}{(}\PY{p}{)}   \PY{c+c1}{\PYZsh{} 왜 여기서 int가 들어가면 안됩니까?}

\PY{n}{a} \PY{o}{=} \PY{n+nb}{int}\PY{p}{(}\PY{n}{a}\PY{p}{[}\PY{p}{:}\PY{p}{:}\PY{o}{\PYZhy{}}\PY{l+m+mi}{1}\PY{p}{]}\PY{p}{)}  \PY{c+c1}{\PYZsh{} [::\PYZhy{}1] 역순}
\PY{n}{b} \PY{o}{=} \PY{n+nb}{int}\PY{p}{(}\PY{n}{b}\PY{p}{[}\PY{p}{:}\PY{p}{:}\PY{o}{\PYZhy{}}\PY{l+m+mi}{1}\PY{p}{]}\PY{p}{)}

\PY{n+nb}{print}\PY{p}{(}\PY{n}{a}\PY{p}{)} \PY{k}{if} \PY{n}{a}\PY{o}{\PYZgt{}}\PY{n}{b} \PY{k}{else} \PY{n+nb}{print}\PY{p}{(}\PY{n}{b}\PY{p}{)}
\end{Verbatim}
\end{tcolorbox}

    \begin{tcolorbox}[breakable, size=fbox, boxrule=1pt, pad at break*=1mm,colback=cellbackground, colframe=cellborder]
\prompt{In}{incolor}{ }{\boxspacing}
\begin{Verbatim}[commandchars=\\\{\}]
\PY{c+c1}{\PYZsh{} 백준 2920 음계}
\PY{c+c1}{\PYZsh{} sort 정렬, 기본값은 오름차순 정렬, reverse옵션 True는 내림차순 정렬}
\PY{c+c1}{\PYZsh{} sort의 key옵션, 지정된 함수의 결과에 따라 정렬}

\PY{n}{a} \PY{o}{=} \PY{n+nb}{list}\PY{p}{(}\PY{n+nb}{map}\PY{p}{(}\PY{n+nb}{int}\PY{p}{,}\PY{n+nb}{input}\PY{p}{(}\PY{p}{)}\PY{o}{.}\PY{n}{split}\PY{p}{(}\PY{p}{)}\PY{p}{)}\PY{p}{)}
\PY{n}{result} \PY{o}{=} \PY{p}{[}\PY{l+m+mi}{1}\PY{p}{,}\PY{l+m+mi}{2}\PY{p}{,}\PY{l+m+mi}{3}\PY{p}{,}\PY{l+m+mi}{4}\PY{p}{,}\PY{l+m+mi}{5}\PY{p}{,}\PY{l+m+mi}{6}\PY{p}{,}\PY{l+m+mi}{7}\PY{p}{,}\PY{l+m+mi}{8}\PY{p}{]}

\PY{k}{if} \PY{n}{a} \PY{o}{==} \PY{n}{result}\PY{p}{:}
    \PY{n+nb}{print} \PY{p}{(}\PY{l+s+s2}{\PYZdq{}}\PY{l+s+s2}{ascending}\PY{l+s+s2}{\PYZdq{}}\PY{p}{)}
\PY{k}{elif} \PY{n}{a} \PY{o}{==} \PY{n}{result}\PY{p}{[}\PY{p}{:}\PY{p}{:}\PY{o}{\PYZhy{}}\PY{l+m+mi}{1}\PY{p}{]}\PY{p}{:}  \PY{c+c1}{\PYZsh{}elif : else if }
    \PY{n+nb}{print}\PY{p}{(}\PY{l+s+s2}{\PYZdq{}}\PY{l+s+s2}{descending}\PY{l+s+s2}{\PYZdq{}}\PY{p}{)}
\PY{k}{else}\PY{p}{:}
    \PY{n+nb}{print}\PY{p}{(}\PY{l+s+s2}{\PYZdq{}}\PY{l+s+s2}{mixed}\PY{l+s+s2}{\PYZdq{}}\PY{p}{)}
    
    
\end{Verbatim}
\end{tcolorbox}

    \begin{tcolorbox}[breakable, size=fbox, boxrule=1pt, pad at break*=1mm,colback=cellbackground, colframe=cellborder]
\prompt{In}{incolor}{ }{\boxspacing}
\begin{Verbatim}[commandchars=\\\{\}]
\PY{c+c1}{\PYZsh{}백준 3052 나머지}

\PY{n}{arr} \PY{o}{=} \PY{p}{[}\PY{p}{]}
\PY{k}{for} \PY{n}{i} \PY{o+ow}{in} \PY{n+nb}{range}\PY{p}{(}\PY{l+m+mi}{10}\PY{p}{)}\PY{p}{:}
    \PY{n}{n}\PY{o}{=} \PY{n+nb}{int}\PY{p}{(}\PY{n+nb}{input}\PY{p}{(}\PY{p}{)}\PY{p}{)}
    \PY{n}{arr}\PY{o}{.}\PY{n}{append}\PY{p}{(}\PY{n}{n}\PY{o}{\PYZpc{}}\PY{k}{42})

\PY{n}{arr} \PY{o}{=} \PY{n+nb}{set}\PY{p}{(}\PY{n}{arr}\PY{p}{)}
\PY{n+nb}{print}\PY{p}{(}\PY{n+nb}{len}\PY{p}{(}\PY{n}{arr}\PY{p}{)}\PY{p}{)}
\end{Verbatim}
\end{tcolorbox}

    \begin{tcolorbox}[breakable, size=fbox, boxrule=1pt, pad at break*=1mm,colback=cellbackground, colframe=cellborder]
\prompt{In}{incolor}{ }{\boxspacing}
\begin{Verbatim}[commandchars=\\\{\}]
\PY{c+c1}{\PYZsh{} 백준 8958 OX}

\PY{n}{a}\PY{o}{=} \PY{n+nb}{int}\PY{p}{(}\PY{n+nb}{input}\PY{p}{(}\PY{p}{)}\PY{p}{)}
\PY{k}{for} \PY{n}{i} \PY{o+ow}{in} \PY{n+nb}{range}\PY{p}{(}\PY{n}{a}\PY{p}{)}\PY{p}{:}
    \PY{n}{b}\PY{o}{=}\PY{n+nb}{input}\PY{p}{(}\PY{p}{)}
    \PY{n}{s}\PY{o}{=}\PY{n+nb}{list}\PY{p}{(}\PY{n}{b}\PY{p}{)}
    \PY{n+nb}{sum}\PY{o}{=}\PY{l+m+mi}{0}
    \PY{n}{c}\PY{o}{=}\PY{l+m+mi}{1}
    \PY{k}{for} \PY{n}{i} \PY{o+ow}{in} \PY{n}{s}\PY{p}{:}
        \PY{k}{if} \PY{n}{i} \PY{o}{==} \PY{l+s+s1}{\PYZsq{}}\PY{l+s+s1}{O}\PY{l+s+s1}{\PYZsq{}}\PY{p}{:}
            \PY{n+nb}{sum} \PY{o}{+}\PY{o}{=} \PY{n}{c}
            \PY{n}{c} \PY{o}{+}\PY{o}{=} \PY{l+m+mi}{1}
        \PY{k}{else}\PY{p}{:}
            \PY{n}{c}\PY{o}{=}\PY{l+m+mi}{1}
    \PY{n+nb}{print} \PY{p}{(}\PY{n+nb}{sum}\PY{p}{)}
\end{Verbatim}
\end{tcolorbox}

    \begin{tcolorbox}[breakable, size=fbox, boxrule=1pt, pad at break*=1mm,colback=cellbackground, colframe=cellborder]
\prompt{In}{incolor}{ }{\boxspacing}
\begin{Verbatim}[commandchars=\\\{\}]
\PY{c+c1}{\PYZsh{} 백준 8958 OX (2)}

\PY{n}{a}\PY{o}{=} \PY{n+nb}{int}\PY{p}{(}\PY{n+nb}{input}\PY{p}{(}\PY{p}{)}\PY{p}{)}
                        \PY{c+c1}{\PYZsh{}처음에 정수 a를 받고 이후 for 문을 a번 만큼 반복}
\PY{k}{for} \PY{n}{i} \PY{o+ow}{in} \PY{n+nb}{range}\PY{p}{(}\PY{n}{a}\PY{p}{)}\PY{p}{:}   
    \PY{n}{b} \PY{o}{=} \PY{n+nb}{list}\PY{p}{(}\PY{n+nb}{input}\PY{p}{(}\PY{p}{)}\PY{p}{)}  
    \PY{n}{score} \PY{o}{=} \PY{l+m+mi}{0}    \PY{c+c1}{\PYZsh{}score 변수를 0으로 지정}
    \PY{n}{sum\PYZus{}score} \PY{o}{=} \PY{l+m+mi}{0}   \PY{c+c1}{\PYZsh{} 새로운 ox리스트를 입력받으면 점수 합계를 리셋.}
    \PY{k}{for} \PY{n}{ox} \PY{o+ow}{in} \PY{n}{b}\PY{p}{:}   
        \PY{k}{if} \PY{n}{ox} \PY{o}{==} \PY{l+s+s1}{\PYZsq{}}\PY{l+s+s1}{O}\PY{l+s+s1}{\PYZsq{}}\PY{p}{:}
            \PY{n}{score} \PY{o}{+}\PY{o}{=} \PY{l+m+mi}{1}   \PY{c+c1}{\PYZsh{} \PYZsq{}O\PYZsq{}가 연속되면 점수가 1씪 커짐}
            \PY{n}{sum\PYZus{}score} \PY{o}{+}\PY{o}{=} \PY{n}{score}
        \PY{k}{else}\PY{p}{:}
            \PY{n}{score} \PY{o}{=} \PY{l+m+mi}{0}
    \PY{n+nb}{print} \PY{p}{(}\PY{n}{sum\PYZus{}score}\PY{p}{)}
\end{Verbatim}
\end{tcolorbox}

    \begin{tcolorbox}[breakable, size=fbox, boxrule=1pt, pad at break*=1mm,colback=cellbackground, colframe=cellborder]
\prompt{In}{incolor}{ }{\boxspacing}
\begin{Verbatim}[commandchars=\\\{\}]
\PY{c+c1}{\PYZsh{}백준 9498 시험성적}

\PY{n}{a} \PY{o}{=} \PY{n+nb}{int}\PY{p}{(}\PY{n+nb}{input}\PY{p}{(}\PY{p}{)}\PY{p}{)}

\PY{k}{if} \PY{n}{a} \PY{o}{\PYZlt{}}\PY{o}{=} \PY{l+m+mi}{100} \PY{o+ow}{and} \PY{n}{a} \PY{o}{\PYZgt{}}\PY{o}{=} \PY{l+m+mi}{90}\PY{p}{:}
    \PY{n+nb}{print} \PY{p}{(}\PY{l+s+s1}{\PYZsq{}}\PY{l+s+s1}{A}\PY{l+s+s1}{\PYZsq{}}\PY{p}{)}


\PY{k}{elif} \PY{n}{a} \PY{o}{\PYZlt{}}\PY{o}{=} \PY{l+m+mi}{89} \PY{o+ow}{and} \PY{n}{a} \PY{o}{\PYZgt{}}\PY{o}{=} \PY{l+m+mi}{80}\PY{p}{:}
    \PY{n+nb}{print} \PY{p}{(}\PY{l+s+s1}{\PYZsq{}}\PY{l+s+s1}{B}\PY{l+s+s1}{\PYZsq{}}\PY{p}{)}


\PY{k}{elif} \PY{n}{a} \PY{o}{\PYZlt{}}\PY{o}{=} \PY{l+m+mi}{79} \PY{o+ow}{and} \PY{n}{a} \PY{o}{\PYZgt{}}\PY{o}{=} \PY{l+m+mi}{70}\PY{p}{:}
    \PY{n+nb}{print} \PY{p}{(}\PY{l+s+s1}{\PYZsq{}}\PY{l+s+s1}{C}\PY{l+s+s1}{\PYZsq{}}\PY{p}{)}
    
\PY{k}{elif} \PY{n}{a} \PY{o}{\PYZlt{}}\PY{o}{=} \PY{l+m+mi}{69} \PY{o+ow}{and} \PY{n}{a} \PY{o}{\PYZgt{}}\PY{o}{=} \PY{l+m+mi}{60}\PY{p}{:}
    \PY{n+nb}{print} \PY{p}{(}\PY{l+s+s1}{\PYZsq{}}\PY{l+s+s1}{D}\PY{l+s+s1}{\PYZsq{}}\PY{p}{)}

\PY{k}{else}\PY{p}{:}
    \PY{n+nb}{print} \PY{p}{(}\PY{l+s+s1}{\PYZsq{}}\PY{l+s+s1}{F}\PY{l+s+s1}{\PYZsq{}}\PY{p}{)}
\end{Verbatim}
\end{tcolorbox}

    \begin{tcolorbox}[breakable, size=fbox, boxrule=1pt, pad at break*=1mm,colback=cellbackground, colframe=cellborder]
\prompt{In}{incolor}{18}{\boxspacing}
\begin{Verbatim}[commandchars=\\\{\}]
\PY{c+c1}{\PYZsh{}백준 10171 고양이}

\PY{n+nb}{print}\PY{p}{(}\PY{l+s+s2}{\PYZdq{}}\PY{l+s+se}{\PYZbs{}\PYZbs{}}\PY{l+s+s2}{    /}\PY{l+s+se}{\PYZbs{}\PYZbs{}}\PY{l+s+s2}{\PYZdq{}}\PY{p}{)}
\PY{n+nb}{print}\PY{p}{(}\PY{l+s+s2}{\PYZdq{}}\PY{l+s+s2}{ )  ( }\PY{l+s+s2}{\PYZsq{}}\PY{l+s+s2}{)}\PY{l+s+s2}{\PYZdq{}}\PY{p}{)}
\PY{n+nb}{print}\PY{p}{(}\PY{l+s+s2}{\PYZdq{}}\PY{l+s+s2}{(  /  )}\PY{l+s+s2}{\PYZdq{}}\PY{p}{)}
\PY{n+nb}{print}\PY{p}{(}\PY{l+s+s2}{\PYZdq{}}\PY{l+s+s2}{ }\PY{l+s+se}{\PYZbs{}\PYZbs{}}\PY{l+s+s2}{(\PYZus{}\PYZus{})|}\PY{l+s+s2}{\PYZdq{}}\PY{p}{)}

\PY{n+nb}{print}\PY{p}{(}\PY{l+s+s2}{\PYZdq{}}\PY{l+s+se}{\PYZbs{}\PYZbs{}}\PY{l+s+s2}{    /}\PY{l+s+se}{\PYZbs{}\PYZbs{}}\PY{l+s+s2}{\PYZdq{}}\PY{p}{)}
\PY{n+nb}{print}\PY{p}{(}\PY{l+s+s2}{\PYZdq{}}\PY{l+s+s2}{ )  ( }\PY{l+s+s2}{\PYZsq{}}\PY{l+s+s2}{)}\PY{l+s+s2}{\PYZdq{}}\PY{p}{)}
\PY{n+nb}{print}\PY{p}{(}\PY{l+s+s2}{\PYZdq{}}\PY{l+s+s2}{(  /  )}\PY{l+s+s2}{\PYZdq{}}\PY{p}{)}
\PY{n+nb}{print}\PY{p}{(}\PY{l+s+s2}{\PYZdq{}}\PY{l+s+s2}{ }\PY{l+s+se}{\PYZbs{}\PYZbs{}}\PY{l+s+s2}{(\PYZus{}\PYZus{})|}\PY{l+s+s2}{\PYZdq{}}\PY{p}{)}
\end{Verbatim}
\end{tcolorbox}

    \begin{Verbatim}[commandchars=\\\{\}]
\textbackslash{}    /\textbackslash{}
 )  ( ')
(  /  )
 \textbackslash{}(\_\_)|
\textbackslash{}    /\textbackslash{}
 )  ( ')
(  /  )
 \textbackslash{}(\_\_)|
    \end{Verbatim}

    \begin{tcolorbox}[breakable, size=fbox, boxrule=1pt, pad at break*=1mm,colback=cellbackground, colframe=cellborder]
\prompt{In}{incolor}{23}{\boxspacing}
\begin{Verbatim}[commandchars=\\\{\}]
\PY{c+c1}{\PYZsh{}백준 10172 개}

\PY{n+nb}{print}\PY{p}{(}\PY{l+s+s2}{\PYZdq{}}\PY{l+s+s2}{|}\PY{l+s+s2}{\PYZbs{}}\PY{l+s+s2}{\PYZus{}/|}\PY{l+s+s2}{\PYZdq{}}\PY{p}{)}
\PY{n+nb}{print}\PY{p}{(}\PY{l+s+s2}{\PYZdq{}}\PY{l+s+s2}{|q p|   /\PYZcb{}}\PY{l+s+s2}{\PYZdq{}}\PY{p}{)}
\PY{n+nb}{print}\PY{p}{(}\PY{l+s+s2}{\PYZdq{}}\PY{l+s+s2}{( 0 )}\PY{l+s+se}{\PYZbs{}\PYZdq{}}\PY{l+s+se}{\PYZbs{}\PYZdq{}}\PY{l+s+se}{\PYZbs{}\PYZdq{}}\PY{l+s+se}{\PYZbs{}\PYZbs{}}\PY{l+s+s2}{\PYZdq{}}\PY{p}{)}
\PY{n+nb}{print}\PY{p}{(}\PY{l+s+s2}{\PYZdq{}}\PY{l+s+s2}{|}\PY{l+s+se}{\PYZbs{}\PYZdq{}}\PY{l+s+s2}{\PYZca{}}\PY{l+s+se}{\PYZbs{}\PYZdq{}}\PY{l+s+s2}{`    |}\PY{l+s+s2}{\PYZdq{}}\PY{p}{)}
\PY{n+nb}{print}\PY{p}{(}\PY{l+s+s2}{\PYZdq{}}\PY{l+s+s2}{||\PYZus{}/=}\PY{l+s+se}{\PYZbs{}\PYZbs{}}\PY{l+s+s2}{\PYZbs{}}\PY{l+s+s2}{\PYZus{}\PYZus{}|}\PY{l+s+s2}{\PYZdq{}}\PY{p}{)}
\end{Verbatim}
\end{tcolorbox}

    \begin{Verbatim}[commandchars=\\\{\}]
|\textbackslash{}\_/|
|q p|   /\}
( 0 )"""\textbackslash{}
|"\^{}"`    |
||\_/=\textbackslash{}\textbackslash{}\_\_|
    \end{Verbatim}

    \begin{tcolorbox}[breakable, size=fbox, boxrule=1pt, pad at break*=1mm,colback=cellbackground, colframe=cellborder]
\prompt{In}{incolor}{ }{\boxspacing}
\begin{Verbatim}[commandchars=\\\{\}]
\PY{c+c1}{\PYZsh{}백준 10809 알파벳 찾기}

\PY{n}{word} \PY{o}{=} \PY{n+nb}{input}\PY{p}{(}\PY{p}{)}
\PY{n}{alphabet} \PY{o}{=}\PY{n+nb}{list}\PY{p}{(}\PY{n+nb}{range}\PY{p}{(}\PY{l+m+mi}{97}\PY{p}{,}\PY{l+m+mi}{123}\PY{p}{)}\PY{p}{)} \PY{c+c1}{\PYZsh{}아스키코드 숫자범위(?) : a =97, z =122}
\PY{c+c1}{\PYZsh{}근데 왜 123까지 함\PYZgt{}?}

\PY{k}{for} \PY{n}{i} \PY{o+ow}{in} \PY{n}{alphabet} \PY{p}{:}
    \PY{n+nb}{print}\PY{p}{(}\PY{n}{word}\PY{o}{.}\PY{n}{find}\PY{p}{(}\PY{n+nb}{chr}\PY{p}{(}\PY{n}{i}\PY{p}{)}\PY{p}{)}\PY{p}{)}   \PY{c+c1}{\PYZsh{}chr : 아스키 코드에 해당하는 숫자를 문자열로 변환}
\end{Verbatim}
\end{tcolorbox}

    \begin{tcolorbox}[breakable, size=fbox, boxrule=1pt, pad at break*=1mm,colback=cellbackground, colframe=cellborder]
\prompt{In}{incolor}{ }{\boxspacing}
\begin{Verbatim}[commandchars=\\\{\}]
\PY{c+c1}{\PYZsh{}백준 10809 알파벳 찾기(2)}

\PY{n}{X} \PY{o}{=} \PY{n+nb}{input}\PY{p}{(}\PY{p}{)}
\PY{n}{abc} \PY{o}{=} \PY{l+s+s1}{\PYZsq{}}\PY{l+s+s1}{abcdefghijklmnopqrstuvwxyz}\PY{l+s+s1}{\PYZsq{}}

\PY{k}{for} \PY{n}{i} \PY{o+ow}{in} \PY{n}{abc}\PY{p}{:}
    \PY{k}{if} \PY{n}{i} \PY{o+ow}{in} \PY{n}{X}\PY{p}{:}    \PY{c+c1}{\PYZsh{}index함수를 통해 X 위치의 알파벳을 반환}
        \PY{n+nb}{print}\PY{p}{(}\PY{n}{X}\PY{o}{.}\PY{n}{index}\PY{p}{(}\PY{n}{i}\PY{p}{)}\PY{p}{,} \PY{n}{end}\PY{o}{=} \PY{l+s+s1}{\PYZsq{}}\PY{l+s+s1}{ }\PY{l+s+s1}{\PYZsq{}}\PY{p}{)} \PY{c+c1}{\PYZsh{}한 칸씩 띄어서 위치 반환 \PYZhy{}\PYZhy{}\PYZgt{}\PYZsq{} \PYZsq{}}
    \PY{k}{else}\PY{p}{:}
        \PY{n+nb}{print}\PY{p}{(} \PY{o}{\PYZhy{}}\PY{l+m+mi}{1}\PY{p}{,} \PY{n}{end} \PY{o}{=} \PY{l+s+s1}{\PYZsq{}}\PY{l+s+s1}{ }\PY{l+s+s1}{\PYZsq{}}\PY{p}{)}   \PY{c+c1}{\PYZsh{}존재하지 않는 알파벳을 \PYZhy{}1 로 반환}
        
        
\end{Verbatim}
\end{tcolorbox}

    \begin{tcolorbox}[breakable, size=fbox, boxrule=1pt, pad at break*=1mm,colback=cellbackground, colframe=cellborder]
\prompt{In}{incolor}{ }{\boxspacing}
\begin{Verbatim}[commandchars=\\\{\}]
\PY{c+c1}{\PYZsh{}백준 10818 최소,최대 }

\PY{n}{a} \PY{o}{=} \PY{n+nb}{int}\PY{p}{(}\PY{n+nb}{input}\PY{p}{(}\PY{p}{)}\PY{p}{)}     \PY{c+c1}{\PYZsh{}정수의 개수 입력 받기}
\PY{n}{b} \PY{o}{=} \PY{n+nb}{list}\PY{p}{(}\PY{n+nb}{map}\PY{p}{(}\PY{n+nb}{int}\PY{p}{,}\PY{n+nb}{input}\PY{p}{(}\PY{p}{)}\PY{o}{.}\PY{n}{split}\PY{p}{(}\PY{p}{)}\PY{p}{)}\PY{p}{)}   \PY{c+c1}{\PYZsh{}정수 입력 받기+input().split()을 통해 공백으로 구분하기}
\PY{n+nb}{max} \PY{o}{=} \PY{n}{b}\PY{p}{[}\PY{l+m+mi}{0}\PY{p}{]}   
\PY{n+nb}{min} \PY{o}{=} \PY{n}{b}\PY{p}{[}\PY{l+m+mi}{0}\PY{p}{]}    \PY{c+c1}{\PYZsh{}각각 첫 번쨰 요소 넣어주기 }

\PY{k}{for} \PY{n}{i} \PY{o+ow}{in} \PY{n}{b}\PY{p}{[}\PY{l+m+mi}{1}\PY{p}{:}\PY{p}{]}\PY{p}{:}    \PY{c+c1}{\PYZsh{}b의 두번째 요소부터 마지막 요소까지 비교}
    \PY{k}{if} \PY{n}{i} \PY{o}{\PYZgt{}} \PY{n+nb}{max} \PY{p}{:}
        \PY{n+nb}{max} \PY{o}{=} \PY{n}{i}        \PY{c+c1}{\PYZsh{}max보다 크면 max 값을 바꿔주고 }
    \PY{k}{elif} \PY{n}{i} \PY{o}{\PYZlt{}} \PY{n+nb}{min}\PY{p}{:}
        \PY{n+nb}{min} \PY{o}{=} \PY{n}{i}     \PY{c+c1}{\PYZsh{}min 보다 작으면 min 값을 바꿔준다. }

        
\PY{n+nb}{print} \PY{p}{(}\PY{n+nb}{min}\PY{p}{,}\PY{n+nb}{max}\PY{p}{)}
\end{Verbatim}
\end{tcolorbox}

    \begin{tcolorbox}[breakable, size=fbox, boxrule=1pt, pad at break*=1mm,colback=cellbackground, colframe=cellborder]
\prompt{In}{incolor}{ }{\boxspacing}
\begin{Verbatim}[commandchars=\\\{\}]
\PY{c+c1}{\PYZsh{}백준 10818 최소,최대 (2)}

\PY{n}{a}\PY{o}{=} \PY{n+nb}{int}\PY{p}{(}\PY{n+nb}{input}\PY{p}{(}\PY{p}{)}\PY{p}{)}
\PY{n}{b} \PY{o}{=} \PY{n+nb}{list}\PY{p}{(}\PY{n+nb}{map}\PY{p}{(}\PY{n+nb}{int}\PY{p}{,}\PY{n+nb}{input}\PY{p}{(}\PY{p}{)}\PY{o}{.}\PY{n}{split}\PY{p}{(}\PY{p}{)}\PY{p}{)}\PY{p}{)}
\PY{n+nb}{print}\PY{p}{(}\PY{n+nb}{min}\PY{p}{(}\PY{n}{b}\PY{p}{)}\PY{p}{,}\PY{n+nb}{max}\PY{p}{(}\PY{n}{b}\PY{p}{)}\PY{p}{)}          \PY{c+c1}{\PYZsh{}그냥 파이썬의 내장함수 min()/max()사용}
\end{Verbatim}
\end{tcolorbox}

    \begin{tcolorbox}[breakable, size=fbox, boxrule=1pt, pad at break*=1mm,colback=cellbackground, colframe=cellborder]
\prompt{In}{incolor}{ }{\boxspacing}
\begin{Verbatim}[commandchars=\\\{\}]
\PY{c+c1}{\PYZsh{}백준 10871 x보다 작은 수 }

\PY{n}{a}\PY{p}{,} \PY{n}{b} \PY{o}{=} \PY{n+nb}{map}\PY{p}{(}\PY{n+nb}{int}\PY{p}{,} \PY{n+nb}{input}\PY{p}{(}\PY{p}{)}\PY{o}{.}\PY{n}{split}\PY{p}{(}\PY{p}{)}\PY{p}{)}
\PY{n}{c} \PY{o}{=} \PY{n+nb}{list}\PY{p}{(}\PY{n+nb}{map}\PY{p}{(}\PY{n+nb}{int}\PY{p}{,} \PY{n+nb}{input}\PY{p}{(}\PY{p}{)}\PY{o}{.}\PY{n}{split}\PY{p}{(}\PY{p}{)}\PY{p}{)}\PY{p}{)}  \PY{c+c1}{\PYZsh{}c에 입력되는 수를 list에 정수로}

\PY{k}{for} \PY{n}{i} \PY{o+ow}{in} \PY{n+nb}{range}\PY{p}{(}\PY{n}{a}\PY{p}{)}\PY{p}{:}
    \PY{k}{if} \PY{n}{c}\PY{p}{[}\PY{n}{i}\PY{p}{]} \PY{o}{\PYZlt{}} \PY{n}{b}\PY{p}{:}
        \PY{n+nb}{print}\PY{p}{(}\PY{n}{c}\PY{p}{[}\PY{n}{i}\PY{p}{]}\PY{p}{,} \PY{n}{end}\PY{o}{=} \PY{l+s+s2}{\PYZdq{}}\PY{l+s+s2}{ }\PY{l+s+s2}{\PYZdq{}}\PY{p}{)}
         
         
\end{Verbatim}
\end{tcolorbox}

    \begin{tcolorbox}[breakable, size=fbox, boxrule=1pt, pad at break*=1mm,colback=cellbackground, colframe=cellborder]
\prompt{In}{incolor}{ }{\boxspacing}
\begin{Verbatim}[commandchars=\\\{\}]
\PY{c+c1}{\PYZsh{}백준 10950 a+b\PYZhy{}3}
\PY{n}{c}\PY{o}{=}\PY{n+nb}{int}\PY{p}{(}\PY{n+nb}{input}\PY{p}{(}\PY{p}{)}\PY{p}{)}
\PY{k}{for} \PY{n}{i} \PY{o+ow}{in} \PY{n+nb}{range}\PY{p}{(}\PY{n}{c}\PY{p}{)}\PY{p}{:}
    \PY{n}{a}\PY{p}{,}\PY{n}{b} \PY{o}{=} \PY{n+nb}{map}\PY{p}{(}\PY{n+nb}{int}\PY{p}{,}\PY{n+nb}{input}\PY{p}{(}\PY{p}{)}\PY{o}{.}\PY{n}{split}\PY{p}{(}\PY{p}{)}\PY{p}{)}
    \PY{n+nb}{print} \PY{p}{(}\PY{n}{a}\PY{o}{+}\PY{n}{b}\PY{p}{)}
\end{Verbatim}
\end{tcolorbox}

    \begin{tcolorbox}[breakable, size=fbox, boxrule=1pt, pad at break*=1mm,colback=cellbackground, colframe=cellborder]
\prompt{In}{incolor}{ }{\boxspacing}
\begin{Verbatim}[commandchars=\\\{\}]
\PY{c+c1}{\PYZsh{}백준 10951 a+b \PYZhy{} 4}

\PY{c+c1}{\PYZsh{}try, except 문}
\PY{c+c1}{\PYZsh{}try블록 수행 중 오류 시 except 블록이 수행 }

\PY{c+c1}{\PYZsh{}while : \PYZti{}\PYZti{}까지 작업을 수행 Or break문을 통해 중지}

\PY{k}{while} \PY{l+m+mi}{1}\PY{p}{:}
    \PY{k}{try}\PY{p}{:}
        \PY{n}{a}\PY{p}{,}\PY{n}{b} \PY{o}{=} \PY{n+nb}{map}\PY{p}{(}\PY{n+nb}{int}\PY{p}{,}\PY{n+nb}{input}\PY{p}{(}\PY{p}{)}\PY{o}{.}\PY{n}{split}\PY{p}{(}\PY{p}{)}\PY{p}{)}
        \PY{n+nb}{print}\PY{p}{(}\PY{n}{a}\PY{o}{+}\PY{n}{b}\PY{p}{)}
    \PY{k}{except}\PY{p}{:}
        \PY{k}{break}
\end{Verbatim}
\end{tcolorbox}

    \begin{tcolorbox}[breakable, size=fbox, boxrule=1pt, pad at break*=1mm,colback=cellbackground, colframe=cellborder]
\prompt{In}{incolor}{ }{\boxspacing}
\begin{Verbatim}[commandchars=\\\{\}]
\PY{c+c1}{\PYZsh{}백준 10951 a+b \PYZhy{} 4}

\PY{k}{try}\PY{p}{:}
    \PY{k}{while} \PY{l+m+mi}{1}\PY{p}{:}
        \PY{n}{a}\PY{p}{,}\PY{n}{b} \PY{o}{=} \PY{n+nb}{map}\PY{p}{(}\PY{n+nb}{int}\PY{p}{,}\PY{n+nb}{input}\PY{p}{(}\PY{p}{)}\PY{o}{.}\PY{n}{split}\PY{p}{(}\PY{p}{)}\PY{p}{)}
        \PY{n+nb}{print} \PY{p}{(}\PY{n}{a}\PY{o}{+}\PY{n}{b}\PY{p}{)}

\PY{k}{except}\PY{p}{:}
    \PY{n}{exit}\PY{p}{(}\PY{p}{)}
    
\end{Verbatim}
\end{tcolorbox}

    \begin{tcolorbox}[breakable, size=fbox, boxrule=1pt, pad at break*=1mm,colback=cellbackground, colframe=cellborder]
\prompt{In}{incolor}{ }{\boxspacing}
\begin{Verbatim}[commandchars=\\\{\}]

\end{Verbatim}
\end{tcolorbox}


    % Add a bibliography block to the postdoc
    
    
    
\end{document}
